\subsection*{Титульная страница по проекту программы для создания графиков }

Пользователь обладает возможностью указывать координаты точек как с помощью ввода значений с клавиатуры, так и с помощью нажатия мышью в определенном месте на главном поле для построения. Все действия отображаются в toolbar главного окна программы. Пользователь имеет возможность добавлять кривые, указав название, цвет, толщину. Но главное – есть возможность построения нескольких кривых на одном графике. Название, цвет кривых указываются в легенде главного поля. Чтобы выбрать с какой кривой работать, пользователь имеет возможность обратиться к одной из созданных в combobox. Также можно удалять кривую, выбрав в Menu\+Bar опцию удаления. Удалять пользователь может и по одной точке. Для этого необходимо выбрать точку для удаления при помощи стрелок перемещения по графику. На главном окне программы присутствует меню для открытия, сохраниние файла. Сохранение файла производится в файлы двух расширений\+: .doc / .txt по выбору. Открытие также реализовано для двух форматов. Для ориентации по документации есть меню слева.~\newline
Основные возможности\+:~\newline
1) создавать управляемые прямые~\newline
2) определять координаты выбранной точки путем нажатия на график~\newline
3) сохранять прямые в файлы разных форматов~\newline
4) загружать прямые из файлов разных форматов~\newline
5) возможность выбрать ближайшую точку активной кривой путем нажатия на график рядом с этой точкой~\newline
6) реализован G\+UI~\newline
~\newline
~\newline
~\newline
~\newline
~\newline
~\newline
~\newline
~\newline
~\newline
~\newline
~\newline
~\newline
~\newline
~\newline
~\newline
~\newline
~\newline
~\newline
 За сим откланяюсь.~\newline
 С уважением, Silence~\newline
